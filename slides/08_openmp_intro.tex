% Document setup
\documentclass[12pt,t]{beamer}
\usetheme[outer/progressbar=none]{metropolis}
% Algorithm environment and layout
\usepackage{algorithmicx}
\usepackage{algpseudocode}
\algloopdefx{Return}{\textbf{return} }
\usepackage{amsmath}
\usepackage{amssymb}
\usepackage{mathtools}
\usepackage[T1]{fontenc}

\usepackage{xparse}
\newbox\FBox
\NewDocumentCommand\Highlight{O{black}O{white}mO{0.5pt}O{0pt}O{0pt}}{%
    \setlength\fboxsep{#4}\sbox\FBox{\fcolorbox{#1}{#2}{#3\rule[-#5]{0pt}{#6}}}\usebox\FBox}

\usepackage{float}
\usepackage{graphicx}
\usepackage[outdir=./]{epstopdf}
\usepackage{color}
\newcommand{\red}[1]{{\color{red}#1}}
\usepackage{caption}
\definecolor{RiceBlue}{HTML}{004080}
\definecolor{BackgroundColor}{rgb}{1.0,1.0,1.0}
\setbeamercolor{frametitle}{bg=RiceBlue}
\setbeamercolor{background canvas}{bg=BackgroundColor}
\setbeamercolor{progress bar}{fg=RiceBlue}
\setbeamertemplate{caption}[default]

\usepackage{multirow}
\usepackage{tabularx}

\usepackage{xcolor}

\usepackage{listings}
\lstset{basicstyle=\footnotesize,showstringspaces=false}

\usepackage{adjustbox}

\usepackage[absolute,overlay]{textpos}

% Footnotes without a number
\newcommand\blfootnote[1]{%
  \begingroup
  \renewcommand\thefootnote{}\footnote{\tiny #1}%
  \addtocounter{footnote}{-1}%
  \endgroup
}

%% Overwrite font settings to make text and math font consistent.
\usepackage[sfdefault,lining]{FiraSans}
\usepackage[slantedGreek]{newtxsf}
\renewcommand*\partial{\textsf{\reflectbox{6}}}
\let\emph\relax % there's no \RedeclareTextFontCommand
\DeclareTextFontCommand{\emph}{\bfseries\em}

\renewcommand*{\vec}[1]{{\boldsymbol{#1}}}

\newcommand{\conclude}[1]{%
  \begin{itemize}
    \item[$\rightarrow$]#1
  \end{itemize}
}
\newcommand{\codeline}[2][]{%
  \begin{lstlisting}[language=c++,#1]^^J
    #2^^J
  \end{lstlisting}
}
\newcommand{\codefile}[2][]{\lstinputlisting[language=c++,frame=single,breaklines=true,#1]{#2}}
\lstnewenvironment{code}{\lstset{language=c++,frame=single}}{}
\newcommand{\cmd}[1]{\begin{center}\texttt{#1}\end{center}}


\begin{document}
  % Title page
  \title{Introduction to OpenMP}
  \subtitle{Computational Science II (CAAM 520)}
  \author{Christopher Thiele}
  \date{Rice University, Spring 2021}

  \setbeamertemplate{footline}{}
  \begin{frame}
    \titlepage
  \end{frame}

  \setbeamertemplate{footline}{
    \usebeamercolor[fg]{page number in head}%
    \usebeamerfont{page number in head}%
    \hspace*{\fill}\footnotesize-\insertframenumber-\hspace*{\fill}
    \vspace*{0.1in}
  }

  \begin{frame}[fragile]
    \frametitle{Overview}

    \begin{itemize}
      \item What is OpenMP?
      \item When to use OpenMP
      \item Using OpenMP in C code
      \item Compiling, linking, and running OpenMP code
      \item The fork-join model
    \end{itemize}
  \end{frame}

  \begin{frame}[fragile]
    \frametitle{What is OpenMP}

    OpenMP (Open Multi-Processing) is an API for multi-processing, mainly multi-threading, in C, C++, and Fortran.

    \begin{itemize}
      \item 1997: OpenMP 1.0 (Fortran)
      \item 1998: First OpenMP standard for C and C++
      \item 2000: OpenMP 2.0 (mainly parallelization of loops)
      \item 2008: OpenMP 3.0 (task parallelism)
      \item 2013: OpenMP 4.0 (accelerators, SIMD)
      \item 2018: OpenMP 5.0
    \end{itemize}
  \end{frame}

  \begin{frame}[fragile]
    \frametitle{When to use OpenMP}

    OpenMP implements shared memory parallelism with multi-threading (and accelerators).

    Hence, it is suitable for
    \begin{itemize}
      \item multi-core CPUs,
      \item multi-CPU systems, and
      \item systems with accelerators.
    \end{itemize}

    It is not suitable for distributed memory environments such as computer clusters.
  \end{frame}

  \begin{frame}[fragile]
    \frametitle{Using OpenMP in C code}

    OpenMP consists of two main components: \emph{preprocessor directives} and a \emph{library} of functions.

    \begin{code}
#include <omp.h>

// ...

{
  // Use an OpenMP preprocessor directive.
  #pragma omp parallel
  {
    // Call an OpenMP library function.
    const int thread_num = omp_get_thread_num();
  }
}
    \end{code}
  \end{frame}

  \begin{frame}[fragile]
    \frametitle{Using OpenMP in C code}

    Why this design?

    Pros:
    \begin{itemize}
      \item Preprocessor directives can be used to annotate and parallelize existing code.
      \item If the compiler does not support OpenMP, it can simply ignore the directives.
    \end{itemize}

    Cons:
    \begin{itemize}
      \item Directives can be somewhat limiting.
      \item Compilers must support OpenMP.
      \item Some compilers are bad at this: Microsoft Visual Studio supports OpenMP 2.0 (2000).
    \end{itemize}
  \end{frame}

  \begin{frame}[fragile]
    \frametitle{Compiling, linking, and running OpenMP code}

    Consider the following example:
    \begin{code}
#pragma omp parallel
{
  const int thread_num = omp_get_thread_num();
  const int num_threads = omp_get_num_threads();
  printf("hello, world from thread %d/%d!\n",
         thread_num, num_threads);
}
    \end{code}

    \conclude{By default, the compiler will ignore the OpenMP directive, and the linker will not find \texttt{omp\_get\_num\_threads()} etc.}
  \end{frame}

  \begin{frame}[fragile]
    \frametitle{Compiling, linking, and running OpenMP code}

    Consider the following example:
    \begin{code}
#pragma omp parallel
{
  const int thread_num = omp_get_thread_num();
  const int num_threads = omp_get_num_threads();
  printf("hello, world from thread %d/%d!\n",
         thread_num, num_threads);
}
    \end{code}

    \conclude{By default, the compiler will ignore the OpenMP directive, and the linker will not find \texttt{omp\_get\_num\_threads()} etc.}
  \end{frame}

  \begin{frame}[fragile]
    \frametitle{Compiling, linking, and running OpenMP code}

    We must instruct the compiler to consider OpenMP directives.

    For GCC, use the \texttt{-fopenmp} flag.

    If compilation and linking are done separately, use the \texttt{-fopenmp} flag for both!
  \end{frame}

  \begin{frame}[fragile]
    \frametitle{Compiling, linking, and running OpenMP code}

    We can run OpenMP applications just like any other application.

    How does OpenMP know how many threads to use?
    \begin{itemize}
      \item The \texttt{OMP\_NUM\_THREADS} environment variable can be used to set the initial number of threads.
      \item The number of threads can be modified using the \texttt{omp\_set\_num\_threads()} function anywhere in the code.
      \item The number of threads can be set in the OpenMP directive, e.g.,\codeline{#pragma omp parallel num_threads(4)}
    \end{itemize}
  \end{frame}

  \begin{frame}[fragile]
    \frametitle{The fork-join model}

    When entering a parallel region, an OpenMP application \emph{forks} into multiple threads.

    All threads \emph{join} with the \emph{master thread} when leaving the parallel region.

    OpenMP will create and finish threads automatically as needed.
  \end{frame}

  \begin{frame}[fragile]
    \frametitle{The fork-join model}

    \begin{figure}
      \centering
      \includegraphics[width=\linewidth]{figures/omp_fork_join.png}
    \end{figure}
    \blfootnote{https://en.wikipedia.org/wiki/Fork\%E2\%80\%93join\_model\#/media/File:Fork\_join.svg}
  \end{frame}

  \begin{frame}[fragile]
    \frametitle{The fork-join model}

    Recall that threads are scheduled and executed \emph{independently} by the OS.
    \conclude{They are not executed in any particular order, in one piece, etc.!}

    Recall that threads are a software concept.
    \conclude{It is possible to run more threads than there are CPU cores (oversubscription).}
  \end{frame}
\end{document}
